\usepackage[utf8]{vietnam}
%----------------------------------------------------------
%\usepackage{vntex}
%Thiết lập lề trang giấy
%\usepackage{anysize}
%\marginsize{3.3cm}{2.2cm}{1.3cm}{2.4cm}{geometry}
%\usepackage{blindtext}
%\usepackage[a4paper,bindingoffset=1cm,left=2.1cm,right=2.5cm,top=2.3cm,bottom=3.2cm,footskip=1.2cm]{geometry}
%\usepackage[a4paper,bindingoffset=1.0cm,left=2.5cm,right=2.5cm,top=2.5cm,bottom=3.0cm,footskip=1.5cm]{geometry}
\usepackage[paperwidth=16cm,paperheight=24cm,bindingoffset=1.0cm,left=1.0cm,right=1.3cm,top=1.75cm,bottom=2.0cm,headsep=0.4cm,footskip=0.6cm]{geometry}
%----------------------------------------------------------
\usepackage{longtable}


%Sử dụng các goi
\usepackage{latexsym,amssymb,amsmath,mathrsfs,amsfonts,amsthm}
\usepackage{array}
\usepackage{tocloft}
\usepackage{etex}
\usepackage{url}
\usepackage{multirow}
\usepackage[all]{xy}
\usepackage{xspace}
\usepackage{stmaryrd}
\usepackage{float}

%Các gói dùng cho thuật toán
\usepackage{algpseudocode}
\usepackage[ruled,vlined,linesnumbered,resetcount,algochapter]{algorithm2e}
\SetAlgoCaptionSeparator{.}
\SetKwProg{Fn}{function}{}{}
\SetKwProg{Pro}{procedure}{}{}

%Định nghĩa các hàm
\SetKwFunction{Depth}{Depth}
\SetKwFunction{Append}{Append}	
\SetKwFunction{Take}{Take}
\SetKwFunction{PrintPath}{PrintPath}
\SetKwFunction{BackwardReasoning}{BackwardReasoning}
\SetKwFunction{Push}{Push}
\SetKwFunction{Pop}{Pop}
\SetKwFunction{Sort}{Sort}
\SetKwFunction{DLS}{DLS}
\SetKwFunction{Resolvable}{Resolvable}
\SetKwFunction{Unresolvable}{Unresolvable}
\SetKwFunction{Leaf}{Leaf}
\SetKwFunction{Unify}{Unify}

\SetAlgorithmName{Thuật toán}{}
%----------------------------------------------------------

%Các gói dùng cho hình ảnh
\usepackage{graphicx}
\usepackage{tikz}
\usetikzlibrary{positioning}
\usetikzlibrary{calc}
\usetikzlibrary{shapes}
%\usepackage[caption=false,font=footnotesize]{subfig}
\usepackage[labelsep=period,format=plain,font=small,textfont=it]{caption}
%\captionsetup[table]{skip=3pt}
\captionsetup[figure]{belowskip=-6pt,aboveskip=8pt}
\captionsetup[table]{belowskip=-3pt,aboveskip=4pt}
%\usepackage{apacite}
%----------------------------------------------------------
\usepackage{xcolor}
\usepackage{color}
\usepackage{colortbl}
\usepackage{titlesec}
\usepackage{hyperref}
\usepackage{enumitem}
\usepackage{tabto}
\usepackage{siunitx}

%----------------------------------------------------------
%Các gói dùng cho code
\usepackage{listings}

\renewcommand{\lstlistingname}{Mã nguồn}

\definecolor{backgroundcolor}{rgb}{0.98,0.98,0.98}

%\definecolor{keywordcolor}{rgb}{1.00,0.40,0.00}
\definecolor{keywordcolor}{rgb}{0.00,0.0,0.00}
\definecolor{commentcolor}{rgb}{0.00,0.60,0.00}
\definecolor{textcolor}{rgb}{0.0,0.00,0.0}
%\definecolor{textcolor}{rgb}{0.58,0.00,0.82}
\definecolor{numberlinecolor}{rgb}{0.2,0.2,0.2}

%\definecolor{backgroundcolor}{rgb}{0.95,0.95,0.95}
\definecolor{keywordcolor}{rgb}{0.00,0.0,0.00}
%\definecolor{commentcolor}{rgb}{0.00,0.0,0.00}
%\definecolor{textcolor}{rgb}{0.0,0.00,0.0}
%\definecolor{numberlinecolor}{rgb}{0.0,0.0,0.0}


%\usepackage{courier}
%\newcommand\mycommfont[1]{\footnotesize\ttfamily\textcolor{black}{#1}}
%\SetCommentSty{mycommfont}

\lstset{frame=single,
	language=XML,
	aboveskip=-1mm,
	belowskip=3mm,
	showstringspaces=false,
	columns=flexible,
	backgroundcolor=\color{backgroundcolor},
	basicstyle={\small\ttfamily},
	captionpos=t,
	%numbers=none,
	numbers=left,
	numbersep=5pt,
	numberstyle={\scriptsize\ttfamily\color{numberlinecolor}},
	keywordstyle=\bfseries\ttfamily\color{keywordcolor},
%	keywordstyle=\bfseries\color{keywordcolor},
	commentstyle=\textit\color{commentcolor},
	stringstyle=\color{textcolor},
	rulecolor=\color{black},
	breaklines=true,
	breakatwhitespace=true,
	%keepspaces=true,
	tabsize=3,
	%setlength{\parskip}{10pt}
	title=\lstname 
}

\captionsetup[lstlisting]{justification=raggedleft,singlelinecheck=off}

%----------------------------------------------------------
%Dùng cho bảng biểu
\usepackage{tabularx}
\usepackage{longtable}
\usepackage{slashbox}
\usepackage{arydshln}
%Lùi đầu dòng cho đoạn đầu tiên của một mục
\usepackage{indentfirst}
%Đánh số cho bài tập
\usepackage{exercise,chngcntr}
\counterwithin{Exercise}{chapter}
\counterwithin{Answer}{chapter}
\def\ExerciseListName{\!\!}
\def\AnswerListName{Lời giải}

%----------------------------------------------------------
%Thiết lập khoảng cách giữa các dòng và các đoạn
\raggedbottom
\newcommand{\itemspace}{2pt}
\setlength{\parindent}{20pt}
\setlength{\parskip}{4pt}
\renewcommand{\baselinestretch}{1.25}
\setlist[itemize,1]{leftmargin=32pt}
\setlist[itemize,2]{leftmargin=25pt}
\setlist[itemize]{topsep=\itemspace, parsep=2pt}
\setlist[enumerate,1]{leftmargin=35pt}
\setlist[enumerate]{topsep=\itemspace, parsep=2pt}
%giáo trình xml và ứng dụng
%----------------------------------------------------------
%Dung cho Header và footer

\newcommand{\theverion}{\fontsize{12}{10} \textsc{}}
%\renewcommand{\thepage}{\fontsize{10}{10} \thepage}

\usepackage{fancyhdr}

%\pagestyle{fancy}
\pagestyle{plain}


	\fancyhead[C]{}
	\fancyhead[RE,LO]{\theverion}

	\fancyhead[LE,RO]{}
%	\fancyfoot[C]{}
%	\fancyfoot[RE,LO]{}
%	\fancyfoot[LE,RO]{}
%	\renewcommand{\headrulewidth}{0.25mm}
%	\renewcommand{\footrulewidth}{0pt}	
	
%	\fancypagestyle{appendix}
%	{
%		\fancyhead[C]{}
%		\fancyhead[RE,LO]{}
%		\fancyhead[LE,RO]{\theverion}
%		\fancyfoot[C]{}
%		\fancyfoot[RE,LO]{}
%		\fancyfoot[LE,RO]{}
%		\renewcommand{\headrulewidth}{1pt}
%		\renewcommand{\footrulewidth}{0pt}	
%	}

%Định nghĩa các các ký hiệu
%-------------------------------------------------------------
\newcommand{\mC}			{\mathcal{C}}
\newcommand{\True}			{\mathsf{True}}
\newcommand{\False}			{\mathsf{False}}
\newcommand{\TrueShort}		{\mathsf{T}}
\newcommand{\FalseShort}	{\mathsf{F}}
\newcommand{\AndOr}			{\mathsf{And/Or}}
\newcommand{\AndOnly}		{\mathsf{And}}
\newcommand{\OrOnly}		{\mathsf{Or}}
\newcommand{\MIN}			{\mathsf{MIN}}
\newcommand{\MAX}			{\mathsf{MAX}}

\renewcommand{\sharp}		{\#}

\newcommand{\HRule}{\rule{\linewidth}{0.6mm}}
\newcommand{\myend}			{\mbox{}\hfill\mbox{{\scriptsize$\!\blacksquare$}}}
%\newcommand{\myend}			{\mbox{}\hfill\mbox{{\scriptsize$\!\square$}}}
\renewcommand{\qedsymbol}	{\myend}
%\newenvironment{sketch}{\noindent{\em Proof sketch.}}{\myend\smallskip}

\newcommand{\semiItem}		{\mbox{- }}
\newcommand{\semiBullet}[1] {\mbox{$\bullet$ {\bf {#1}}}}
\newcommand{\tuple}[1]		{\left\langle#1\right\rangle\!}
\newcommand{\ramka}[1]		{\fbox{\parbox{\textwidth}{#1}}}
\newcommand{\V}				{\forall}
\newcommand{\E}				{\exists}


%-------------------------------------------------------------
%Các ký hiệu cho ví dụ (cá thể)
\newcommand{\iAN}			{\mathsf{AN}}
\newcommand{\iBA}			{\mathsf{BA}}
\newcommand{\iBINH}			{\mathsf{BINH}}
\newcommand{\iTU}			{\mathsf{TU}}
\newcommand{\iHOA}			{\mathsf{HOA}}
\newcommand{\iLAN}			{\mathsf{LAN}}
\newcommand{\iROSE}			{\mathsf{ROSE}}
\newcommand{\iFOOTBALL}		{\mathsf{FOOTBALL}}
\newcommand{\iGRAY}			{\mathsf{GRAY}}
\newcommand{\iTENNIS}		{\mathsf{TENNIS}}
\newcommand{\iMUSIC}		{\mathsf{MUSIC}}
\newcommand{\iBIBI}			{\mathsf{BIBI}}
\newcommand{\iROCKY}		{\mathsf{ROCKY}}
%Vị từ
\newcommand{\iFather}		{\mathsf{father}}
\newcommand{\iBrother}		{\mathsf{brother}}
\newcommand{\iMother}		{\mathsf{mother}}
\newcommand{\iPerson}		{\mathsf{person}}
\newcommand{\iParent}		{\mathsf{parent}}
\newcommand{\iHusband}		{\mathsf{husband}}
\newcommand{\iLove}			{\mathsf{love}}
\newcommand{\iPlay}			{\mathsf{play}}
\newcommand{\iLike}			{\mathsf{like}}
\newcommand{\iMale}			{\mathsf{male}}
\newcommand{\iFemale}		{\mathsf{female}}
\newcommand{\iYounger}		{\mathsf{younger}}
\newcommand{\iGood}			{\mathsf{good}}
\newcommand{\iHear}			{\mathsf{hear}}
\newcommand{\iStudy}		{\mathsf{study}}
\newcommand{\iKnow}			{\mathsf{know}}
\newcommand{\iPrime}		{\mathsf{prime}}
\newcommand{\iOdd}			{\mathsf{odd}}
\newcommand{\iFly}			{\mathsf{fly}}
\newcommand{\iRain}			{\mathsf{rain}}
\newcommand{\iDog}			{\mathsf{dog}}
\newcommand{\iCat}			{\mathsf{cat}}
\newcommand{\iRear}			{\mathsf{rear}}
\newcommand{\iKill}			{\mathsf{kill}}
\newcommand{\iAnimalLover}	{\mathsf{animalLover}}
\newcommand{\iAnimal}		{\mathsf{animal}}


\newcommand{\iGreater}		{\mathsf{greater}}
\newcommand{\iLesser}		{\mathsf{lesser}}
\newcommand{\iFriend}		{\mathsf{friend}}
\newcommand{\iBadPerson}	{\mathsf{badPerson}}
\newcommand{\iElephant}		{\mathsf{elephant}}
\newcommand{\iColor}		{\mathsf{color}}
\newcommand{\iStudent}		{\mathsf{student}}
\newcommand{\iInside}		{\mathsf{inside}}

%----------------------------------------------------------
%Định dạng chương, định lý, mệnh đề, ...
\newtheorem{Theorem}{Định lý}[chapter]
\newtheorem{Proposition}{Mệnh đề}[chapter]
\newtheorem{Lemma}{Bổ đề}[chapter]
\newtheorem{Corollary}{Hệ quả}[chapter]
\newtheorem{Remark}{Ghi chú}[chapter]
\newtheorem{Result}{Kết quả}[chapter]

\theoremstyle{definition}
\newtheorem{Definition}{Định nghĩa}[chapter]
\newtheorem{Example}{Ví dụ}[chapter]

%Định nghĩa các màu dùng cho các tiêu đề CHƯƠNG, MỤC
\definecolor{colorChapter}{rgb}{0.00, 0.00, 0.00}
\definecolor{colorSection}{rgb}{0.00, 0.00, 0.00}
%\definecolor{colorChapter}{rgb}{0.81, 0.09, 0.13}
%\definecolor{colorSection}{rgb}{0.16, 0.32, 0.95}
%-----------------------------------------------------------------
%Thiết lập các font cho tiêu đề CHƯƠNG, MỤC, TIỂU MỤC
\newcommand{\chapterFont}{\normalfont\fontsize{14}{15}\bfseries\color{colorChapter}}
\newcommand{\sectionFont}{\normalfont\fontsize{12}{15}\bfseries\color{colorSection}}
\newcommand{\subsectionFont}{\normalfont\fontsize{12}{15}\bfseries\itshape\color{black}}
\newcommand{\subsubsectionFont}{\normalfont\fontsize{12}{15}\bfseries\color{black}}
\renewcommand{\thesubsubsection}{\alph{subsubsection}}
%-----------------------------------------------------------------
%Thiết lập thông số các tiêu đề của CHƯƠNG, MỤC
\setcounter{secnumdepth}{3}

\titleformat{\chapter}[display]{\chapterFont}{\chaptertitlename\ \thechapter}{10pt}{\centering} % display -> block/hange

%\titleformat{\chapter}[display]{\chapterFont}{\chaptertitlename\ \thechapter.}{10pt}{\centering} % display -> block/hange

\titlespacing{\chapter}{0pt}{0pt}{25pt} % left, before, after

\titleformat{\section}{\sectionFont}{\thesection.}{10pt}{}
\titlespacing{\section}{0pt}{6pt}{2pt}

\titleformat{\subsection}{\subsectionFont}{\thesubsection.}{10pt}{}
\titlespacing{\subsection}{0pt}{5pt}{2pt}

\titleformat{\subsubsection}{\subsubsectionFont}{\thesubsubsection.}{10pt}{}
\titlespacing{\subsubsection}{0pt}{5pt}{2pt}

%\renewcommand{\thesubsubsection}{\alph{subsubsection}}
%-----------------------------------------------------------------
%\usepackage{draftwatermark}
%\SetWatermarkText{\;\;\;Version Seminar}
%\SetWatermarkLightness{0.9}
%\SetWatermarkScale{0.55}
%-----------------------------------------------------------------
%Thiết lập thông số mục lục cho CHƯƠNG
\newlength\myLenChap
\renewcommand\cftchappresnum{\chaptername~}
\renewcommand\cftchapaftersnum{.}
\settowidth\myLenChap{\cftchappresnum\cftchapaftersnum}
\addtolength\cftchapnumwidth{\myLenChap}
%-----------------------------------------------------------------
%Thiết lập thông số mục lục cho MỤC
\newlength\myLenSec
\renewcommand\cftsecaftersnum{.}
\settowidth\myLenSec{\cftsecpresnum\!}
\addtolength\cftsecnumwidth{\myLenSec}
%-----------------------------------------------------------------
%Thiết lập thông số mục lục cho TIỂU MỤC
\newlength\myLenSubsec
\renewcommand\cftsubsecaftersnum{.}
\settowidth\myLenSubsec{\cftsubsecpresnum\!\!}
\addtolength\cftsubsecnumwidth{\myLenSubsec}
%-----------------------------------------------------------------
%Thiết lập thông số mục lục cho HÌNH ẢNH
\setlength{\cftfigindent}{0pt}
\newlength{\myLenFig}
\renewcommand{\cftfigpresnum}{\figurename\;}
\renewcommand{\cftfigaftersnum}{.}
\settowidth{\myLenFig}{\cftfigpresnum\!\!}
\addtolength{\cftfignumwidth}{\myLenFig}
%-----------------------------------------------------------------
%Thiết lập thông số mục lục cho BẢNG BIỂU
\setlength{\cfttabindent}{0pt}
\newlength{\myLenTab}
\renewcommand{\cfttabpresnum}{\tablename\;}
\renewcommand{\cfttabaftersnum}{.}
\settowidth{\myLenTab}{\cfttabpresnum\!}
\addtolength{\cfttabnumwidth}{\myLenTab}

\makeatletter
\def\cleardoublepage
{
	\clearpage\if@twoside \ifodd\c@page\else
	\hbox{}
%	\vspace*{\fill}
%	\begin{center}
%		{\em This page intentionally left blank}
%	\end{center}
	\vspace{\fill}
	\begin{flushleft}
		%\noindent\rule{6.0cm}{1pt}\\[-0.2cm]
		%\theverion
	\end{flushleft}
	\thispagestyle{empty}
	\newpage
	\if@twocolumn\hbox{}\newpage\fi\fi\fi
}
\makeatother