\begin{titlepage}
%\begin{tikzpicture}[remember picture, overlay]
%  \draw[line width = 4pt] ($(current page.north west) + (2.5cm,-2cm)$) rectangle ($(current page.south east) + (-1.5cm, 2cm)$);
%  \draw[line width = 3pt] ($(current page.north west) + (0cm,0cm)$) rectangle ($(current page.south east) + (0cm, 0cm)$);
%\end{tikzpicture}

\begin{center}
\vspace{-2.5cm}
\textsc{\fontsize{11.5}{12} \textbf{ĐẠI HỌC HUẾ}}\\
\vspace{-0.05cm}
{\fontsize{11.5}{12} \textbf{TRƯỜNG ĐẠI HỌC KHOA HỌC}}\\
\vspace{-0.45cm}
\rule{7.0cm}{0.25mm}\\[1.8cm]

\textsc{\fontsize{11}{12} \textbf{NGUYỄN HOÀNG HÀ, NGUYỄN VĂN TRUNG}}\\
\vspace{-0.05cm}
\textsc{\fontsize{11}{12} \textbf{ NGUYỄN DŨNG, NGUYỄN MẬU HÂN}}\\
\vspace{-0.05cm}


\vspace{5.0cm}
% Title
%\HRule \\[0.4cm]
{\fontsize{18}{15} \bfseries GIÁO TRÌNH}\\[0.2cm]
{\fontsize{22}{15} \bfseries JAVA CƠ BẢN}\\[2.1cm]
%\HRule \\[4.5cm]

\begin{minipage}{0.45\textwidth}
	\begin{flushleft}
		~
	\end{flushleft}
\end{minipage}
\begin{minipage}{0.50\textwidth}
	\begin{flushright}
		%\textsc{Version $\pi$}
	\end{flushright}
\end{minipage}

\vfill
% Bottom of the page
{\fontsize{11.5}{15} \bf NHÀ XUẤT BẢN ĐẠI HỌC HUẾ}\\
\vspace{-0.05cm}
{\fontsize{12}{15} \bf Huế, 2020}\\
\vspace{-0.6cm}
\end{center}
\end{titlepage}

\newpage
\thispagestyle{empty}

~

\vspace{4cm}
%\noindent
\fontsize{11}{13}\selectfont
{\bf Biên mục trên xuất bản phẩm của Thư viện Quốc gia Việt Nam}\\[0.5cm]
\fontsize{12}{13}\selectfont
%Nguyễn Mậu Hân

%Giáo trình XML và ứng dụng/Nguyễn Mậu Hân(ch.b), Nguyễn Dũng, Nguyễn Hoàng Hà. - Huế: Đại học Huế, 2019. - 211tr.; 25cm

ĐTTS ghi: Đại học Huế. Trường Đại học Khoa học. - Thư mục: tr. 211

\indent %1. Ngôn ngữ XML  2. Ngôn ngữ đánh dấu  3. Ứng dụng  4. Giáo trình

\indent \indent 006.740711 - dc23



\begin{center}
DUF0253p - CIP\begin{center}

\end{center}
\end{center}
\vfill
\begin{center}
\rule{5.2cm}{0.25mm}\\
{\bf Mã số sách: GT/38-2019}
\end{center}

\fontsize{12}{13}\selectfont
\cleardoublepage
%-----------------------------------------------------------------------------

\pagenumbering{roman}
%-----------------------------------------------------------------------------
\chapter*{LỜI NÓI ĐẦU\thispagestyle{fancy}}
\addcontentsline{toc}{chapter}{Lời nói đầu}
\thispagestyle{plain}
%\pagestyle{empty}

%XML viết tắt của eXtensible Markup Language, một ngôn ngữ đánh dấu như HTML, XML được kế thừa từ SGML (Standard Generalized Markup Language), một ngôn ngữ đánh dấu tiêu chuẩn tổng quát, biểu thị dữ liệu trong những ứng dụng xử lý văn bản đa dạng và có cấu trúc tinh vi nhưng nhược điểm của nó là rất rắc rối, khó học, khó sử dụng. XML có vai trò quan trọng trong việc chuyển tải, trao đổi dữ liệu và liên lạc giữa các ứng dụng, đặc biệt các ứng dụng trên nền tảng web.
Giới thiệu về NNLT Java: ....
%XML khá giống HTML, hai ngôn ngữ này có cùng luật cú pháp, cả hai đều được lưu trữ dưới dạng các tập tin văn bản với phần mở rộng lần lượt là html và xml. Tuy nhiên, tính linh hoạt của XML cho phép tạo và sử dụng tập thẻ, thuộc tính riêng để nhận biết các phần tử cấu trúc và nội dung tài liệu. XML không chỉ đơn thuần là ngôn ngữ đánh dấu, mà còn là phương pháp định ra nội dung tài liệu, tương tự như HTML định hình thức tài liệu trên web. Với HTML, người thiết kế đánh dấu văn bản, hình ảnh cùng các thành phần khác của trang web bằng tập thẻ mà không liên quan gì tới ý nghĩa tài liệu; đoạn mã HTML chỉ nói lên cách hiển thị nội dung tài liệu qua trình duyệt. Trong khi đó XML không chỉ định ra hình thức mà còn cả nội dung tài liệu, nó có thể tự mô tả dữ liệu và có thể chứa dữ liệu, nhưng lại không có mục đích hiển thị dữ liệu như HTML. XML là một ngôn ngữ có nền tảng độc lập và được cộng đồng hỗ trợ. Dữ liệu trong XML không được định nghĩa trước, chúng ta  phải tự định nghĩa các thẻ trong tài liệu XML.

%Có thể nói HTML là ngôn ngữ đánh dấu thành công nhất từ trước tới nay. Chúng ta có thể thấy dấu ấn của HTML đơn giản nhất trên bất cứ công cụ nào có sử dụng trên môi trường web, từ thiết bị cầm tay tới máy chủ, thậm chí chúng ta còn có thể chuyển đổi đánh dấu một trang HTML sang lời nói hoặc các định dạng khác với những công cụ chính xác. Một câu hỏi đặt ra là tại sao HTML thành công như thế mà W3C lại tạo ra XML? Người ta nhận thấy rằng HTML chỉ để hiển thị nội dung của một trang web theo một số nguyên tắc có sẵn nào đó. Trong khi các ứng dụng và dịch vụ web (web services) ngày càng phát triển, chúng cần chuyển tải, lưu trữ và trao đổi dữ liệu với nhau, điều mà HTML không làm được. 

Giáo trình “Java cơ bản” sẽ hệ thống toàn bộ những nội dung kiến thức cơ bản liên quan đến kiến thức về ngôn ngữ lập trình Java. Giáo trình được sử dụng cho sinh viên chuyên ngành Công nghệ Thông tin, là môn cơ sở để lập trình di động, lập trình web JSP, ... Giáo trình được chia làm 3 chương. Cuối mỗi chương sẽ có phần tổng kết chương, giúp người học nắm được các nội dung và ý nghĩa của chương. Phần câu hỏi và bài tập ở cuối mỗi chương cũng giúp người học nâng cao kỹ năng thực hành trên cơ sở phần lý thuyết được trang bị. 

Các chương của giáo trình được tóm tắt như sau:

Chương 1: \textit{Lập trình với Java},....

Chương 2: \textit{Lập trình hướng đối tượng trong Java}, ....

Chương 3:\textit{ Lập trình ứng dụng csdl}. ...

Mặc dù đã rất cố gắng để hoàn thiện giáo trình này với mong muốn đến tay người đọc nhưng giáo trình không thể tránh khỏi những thiếu sót về cách diễn đạt, bố cục, nội dung và các lỗi cú pháp. Rất mong được bạn đọc góp ý. 

Để hoàn tất giáo trình này chúng tôi xin cám ơn các Thầy Cô giáo của Khoa Công nghệ Thông tin, Trường Đại học Khoa học, Đại học Huế đã góp ý, chỉnh sửa để giáo trình sớm được ra mắt bạn đọc.   

\begin{flushright}
\textit{Huế, tháng 02 năm 2019}
 
\textbf{Nhóm tác giả }
\end{flushright}

%Mọi đóng góp của quý vị xin gửi về theo địa chỉ:
\vspace{-0.4cm}
%\begin{flushright}
%{\bf Nguyễn Mậu Hân},\\
%Khoa Công nghệ thông tin, Trường Đại học Khoa học\\
%Địa chỉ: 77 Nguyễn Huệ, Thành phố Huế, Tỉnh Thừa Thiên Huế\\
%Email: nmhan@hueuni.edu.vn
%\end{flushright}


%\begin{tabular}{p{10cm} c}
%	& Tập thể tác giả\\
%	& \\
%	& \\
%	& \\
%	& \\
%\end{tabular}
\cleardoublepage
%-----------------------------------------------------------------------------
\newpage
\setcounter{tocdepth}{2}
\addcontentsline{toc}{chapter}{Mục lục}
\renewcommand\contentsname{\!\!\!\!\!\!\!\!\!\!\chapterFont\hfill MỤC LỤC\hfill}
\thispagestyle{empty}
%\pagestyle{empty}

\setlength\cftparskip{-2pt}
\setlength\cftbeforechapskip{1pt}
\tableofcontents{\thispagestyle{fancy}}
\cleardoublepage
%-----------------------------------------------------------------------------
%\newpage
%\chapter*{DANH MỤC TỪ VIẾT TẮT}
%\addcontentsline{toc}{chapter}{Danh mục từ viết tắt}
%\thispagestyle{fancy}
%
%\begin{tabular}{| p{2.5cm} | p{10.0cm} |}
%	\hline
%	\multicolumn{1}{|c|}{\bf Từ viết tắt} & \multicolumn{1}{c|}{\bf Diễn giải}\\
%	\hline
%	\ldots & \ldots\\
%	\hline
%\end{tabular}
%\cleardoublepage
%%-----------------------------------------------------------------------------
%\newpage
%\chapter*{DANH MỤC CÁC KÝ HIỆU}
%\addcontentsline{toc}{chapter}{Danh mục các ký hiệu}
%\thispagestyle{fancy}
%
%\begin{tabular}{| p{2.5cm} | p{10.0cm} |}
%	\hline
%	\multicolumn{1}{|c|}{\bf Ký hiệu} & \multicolumn{1}{c|}{\bf Diễn giải ý nghĩa}\\
%	\hline
%	\ldots & \ldots\\	
%	\hline
%\end{tabular}
%\cleardoublepage
%-----------------------------------------------------------------------------
\renewcommand\listfigurename{\!\!\!\!\!\!\!\!\!\chapterFont\hfill DANH MỤC HÌNH VẼ\hfill}
\addcontentsline{toc}{chapter}{Danh mục hình vẽ}
\listoffigures{\thispagestyle{fancy}}
\cleardoublepage

%-----------------------------------------------------------------------------
\renewcommand\listtablename{\!\!\!\!\!\!\!\!\!\chapterFont\hfill DANH MỤC BẢNG BIỂU\hfill}
\addcontentsline{toc}{chapter}{Danh mục bảng biểu}
\listoftables{\thispagestyle{fancy}}
\cleardoublepage